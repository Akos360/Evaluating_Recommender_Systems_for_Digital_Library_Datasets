\documentclass[10pt,oneside,english,a4paper]{article} 
\usepackage{graphicx}
\usepackage{url} 
\usepackage{doi} 
\usepackage{hyperref}
\usepackage{cite}
\usepackage{comment} 
\usepackage[utf8]{inputenc}

%\usepackage[backend=biber,style=numeric]{biblatex}
%\usepackage[sorting=none]{biblatex} %\addbibresource{literature.bib}
%\DeclareFieldFormat{doi}{doi:\texttt{#1}} %\renewcommand*{\doi}[1]{doi:\texttt{#1}}
%\bibliography{literature}

%%%%%%%%%%%%%%%%%%%%%%%%%%%%%%%%%%%%%%%%%%%%%%%%%%%

% Slovenská technická univerzita v Bratislave Fakulta informatiky a informačných technológií
% Bakalárska práca

\title{\Large \textbf{Slovak University of Technology in Bratislava\\
Faculty of Informatics and Information Technologies} \ \\ \vspace{4\baselineskip} \Large{ Ákos
Lévárdy } \ \\ \LARGE{ Odporúčacie systémy založené na AI } \ \\ \ \\ \large{
Bachelor thesis} }

%\author{Ákos Lévárdy}

%%%%%%%%%%%%%%%%%%%%%%%%%%%%%%%%%%%%%%%%%%%%%%%%%%%
\begin{document} 
\maketitle
\thispagestyle{empty}
\vspace{7\baselineskip} 		% vertical space 
\hspace{-2cm} 			% horizontal position
\parbox{0.8\textwidth}{ 
\raggedright 			% Left-align the text

\Large Degree course: Informatics\\ 
Field of study: 9.2.1 Informatics\\ 
Place: FIIT STU, Bratislava\\ 
Supervisor: PaedDr. Pavol Baťalík \\}

%\Large Študijný program: Informatika\\ 
%Študijný odbor: 9.2.1 Informatika\\ 
%Miesto vypracovania: FIIT STU, Bratislava\\ 
%Vedúci práce: PaedDr. Pavol Baťalík \\}

%%%%%%%%%%%%%%%%%%%%%%%%%%%%%%%%%%%%%%%%%%%%%%%%%%%

 %\today  
\pagenumbering{roman}
\newpage{} 
\setcounter{page}{2}

\hspace{-2cm} \Large \textbf{ANNOTATION}\\
500 words

\newpage{}
\setcounter{page}{3}

\hspace{-2cm} \Large \textbf{ANOTÁCIA}\\ 
500 slov 


\newpage{} 		% Čestné vyhlásenie
\setcounter{page}{4}
\vspace*{\fill}
\noindent \Large \textbf{DECLARATION OF OATH}\\
\noindent I hereby declare upon my honour that I wrote this thesis single-handed with usage of quoted literature and based on my knowledge and professional supervision of my supervisor.
\vspace*{\fill} 
\vspace{-8cm} 



\newpage 			% Poďakovanie
\setcounter{page}{5}
\vspace*{\fill} 
\noindent \Large \textbf{ ACKNOWLEDGMENT}\\
\noindent First and foremost, I would like to thank my supervisor for their invaluable guidance and support throughout the duration of this project.
\vspace*{\fill} 
\vspace{-8cm} 




%%%%%%%%%%%%%%%%%%%%%%%%%%%%%%%%%%%%%%%%%%%%%%%%%%%

\newpage{} 

\setcounter{page}{6}
\tableofcontents
%%%%%%%%%%%%%%%%%%%%%%%%%%%%%%%%%%%%%%%%%%%%%%%%%%%
\newpage{}
\listoffigures
\listoftables
\section*{List of Abbreviations}
%%%%%%%%%%%%%%%%%%%%%%%%%%%%%%%%%%%%%%%%%%%%%%%%%%%


\clearpage{} 
\pagenumbering{arabic}
\setcounter{page}{1}

\section{Introduction}
As Internet and Web technologies continue to evolve rapidly, the amount of information available online has expanded excessively across sections such as e-commerce, e-government, and e-learning. To help users navigate this vast sea of content, Recommender Systems have become fundamental. They are very effective tools for filtering out the most appropriate information any user would like to find. The primary focus of these recommendations is to predict if a specific user will be interested in the distinct items.\\\\
The main target of this project is to create a recommendation system that uses ............. (text, materials).






\clearpage{}
\section{Analysis}
Making decisions is not always easy. People are frequently presented with an overwhelming number of options when picking a product, a movie, or a destination to travel to, and each option comes with different levels of information and trustworthiness. \\
When a user is trying to find a movie to watch, it would be hard for them to start searching without any starting options. After all a blank page and no suggestions to choose from might even make the user decide not to pick anything. \\ 
Because of this amount of detail from all of the items, recommendation systems are becoming increasingly important. They help reduce options and offer better suggestions for the user so that they will have a personalized list to select their favourite. The Recommendation systems provide users with well chosen options for products that fit their requirements and interests, sometimes even matching their tastes.\\\\
Different types of recommendation systems exist, and their methods of operation vary. These recommendation types are divided into 3 different categories, which are Content-Based Filtering approaches (CB), Collaborative Filtering approaches (CF) and Hybrid approaches which are the combinations of the two.\\
Content-Based Filtering works in a way that it creates user profiles and suggests the individual items or products based on the users past choices with similar items. The items have various features and characteristics which connect them.
Collaborative Filtering relies more on preferences of other users and their behaviour. The point is that users who had similar interests before will have them again in the future for new items.\\\\
Both CB and CF approaches encounter significant challenges such as the Cold-Start Problem, Data Sparsity or Scalability. The Cold-Start Problem arises when making recommendations to new users and/or items for which the available information is limited. As a result, the recommendations offered in such cases tend to be of poor quality and lack usefulness.\cite{Al-Hassan2024a}\\\\
%




\clearpage
\subsection{Collaborative Filtering}
One of the most popular methods used for personalized recommendations is collaborative filtering. This method filters information from users, which means it compares users behaviour, interactions with items and data, item correlation and ratings from users. \\\\
Collaborative filtering can be divided into 2 methods which are "Memory-based" and "Model-Based" collaborative filtering. The first one relies on historical preferences, whereas the second method is based on machine learning models to predict the best options.\\\\
There are 2 basic types of memory-based collaborative filtering which are:
\begin{itemize}
\item User-Based Collaborative Filtering
	\begin{itemize}
	\item The main idea is that 2 completely distinct users who have an interest in a specific item and they rate this item similarly will probably be drawn to a new item the same way.
	\end{itemize}
\item Item-Based Collaborative Filtering
	\begin{itemize}
	\item Calculates similarity between items, rather than users. The user will probably like a new item which is similar to another item they were interested in before.\\
	\end{itemize}
\end{itemize}
%
%
%
\clearpage{}
- POZ\\
- analyzes relationships between users  and interdependencies among products to identify new user-item associations\\
- recommend items or content to users by analyzing their interactions and similarities with other users\\\\
\cite{5197422}\\\\
%
- long-tailed users - user groups with relatively uncommon or more diverse interests or preferences\\
- association rules\\
\cite{Yan2024}\\\\


\subsection{Content-Based Filtering}
Recommender Systems which are using content-based filtering, review a variety of items, documents and their details. Each product has their own description which is collected to make a profile for each item. These profiles for items are different representations of information and users interest about the specific item. \\
The recommendation process basically consists in matching
up the attributes of the user profile against the attributes of a content object.
\cite{pub.1034486657}\\
The process for recommending items using content-based filtering has 3 different phases.
\begin{itemize}
\item Content Analyzer
\item Profile Learner
\item Filtering Component
\end{itemize}

\subsection{Feedback}
When trying to ac	quire feedback from the user there are 2 separate ways. The first one is the Explicit Feedback where it is necessary for the user to give item evaluation or actively rate products. Most popular options are gathering like/dislike ratings on items or the ratings can be on a scale either from 1 to 5 or 1 to 10. After the ratings the user can also give comments on separate items. \\
The other way is Implicit Feedback where the information is collected passively from analyzing the users activities. Some alternatives can be clicks on products, time spent on sites or even transaction history.\\



\subsection{Hybrid approach}
- uses both the CF and the CB filtering, for more accuracy




%%%%%%%%%%%%%%%%%%%%%%%%%%%%%%%%%%%%%%%%%%%%%%
\clearpage
\subsection{Difficulties related to recommendation systems}
- cold-start problem\\
- data sparsity\\
- scalability\\
- bias and diversity\\
- ...\\\\
- Exploration VS Exploitation\\\\


\clearpage
\subsection{Performance - measures - Metrics}
- recall rate\\
- root mean square error\\
- precision\\
- cumulative gain ?\\
- accuracy\\
- overall efficacy\\
- f1 - measure\\
- ...\\


\clearpage
\subsection{Semantics ?}

\subsection{Ontology ?}

\subsection{Matrix Factorization}
Matrix factorization (MF) is a technique utilized in collaborative filtering to decompose
a matrix of user-item ratings into lower-rank matrices capturing the latent factors under-
lying the data\cite{Tokala2023}.\\
People prefer to rate just a small percentage of items, therefore the user-item rating matrix, that tracks the ratings people assign to various items, is frequently sparse.\\
In order to deal with this sparsity, matrix factorization (MF) algorithms split the matrix into two lower-rank matrices: one that shows the latent properties of the items and another that reflects the underlying user preferences. These latent representations can be used to predict future ratings or complete the matrix's missing ratings after factorization.

\subsection{Search Engines}
Search Engines have become crucial for navigating the vast amount of information available online. They make it possible for people to quickly look up solutions, learn new things, and browse the wide variety of resources available on the internet. Search engine optimization is now necessary to guarantee that search engines deliver relevant results, quick search times, and a top-notch user experience given the explosive growth of online information.\\
A search engine is essentially a software that finds the information the user needs using keywords or phrases. It delivers results rapidly, even with millions of websites available online.
The importance of speed in online searches is highlighted by how even minor delays in retrieval can negatively affect users' perception of result quality.
\cite{pub.1171882357}

\clearpage
\subsection{Concept Drift ?}
Concept Drift - Entropy-based\\\\
Information systems inevitably experience frequent data changes. 
This change in the statistical properties of the target variable, caused by
unforeseeable variations in the underlying distribution of the data
stream, is known as concept drift.\cite{Sun2024}



%%%%%%%%%%%%%%%%%%%%%%%%%%%%%%%%%%%%%%%%%%%%%%%%%%%%
%
\clearpage{}
POZNAMKY\\
- user confidence\\
- time context\\\\
- personalized -vs- not personalized(recommend based on huge amount of people) recommendations\\
- recommend specific subject, study material, field of study, examples, lessons\\
- "At first it is important to describe / to outline / to define ..."\\
%
%
- Material difficulty level - features\\
- Comments, reviews, ratings, number of views, content, inquiries, and other factors can be used to automatically assign a difficulty level.\\\\
- four primary stages: student profiling, material collection, material filtering, and material validation. \\
- Machine Learning (ML), Decision Making (DM) approaches\\
- Each stage is explained in detail in the sub-sections below.\\
- The DM techniques were used to extract keywords from the material provided to develop queries\\\\
- MATERIALS: textbooks, lecture notes, additional questions, quizzes, exam samples, reports, articles, and books\\\\
%
I. The content-based module: The module is responsible for analyzing the con-
tents of the materials and representing each material with a set of keywords and
assigning them to topics and courses.
II. The collaborative module: The module used the ratings, reviews, and number of
views of the materials in the student’s history.
III. The contextual module: The module used the students’ marks and level of
performance.
IV. The serendipity module: The module used the publicity of the materials and their
reviews in the material database.
\cite{Zayet20237487}\\\\
%
CLASSIFICATION METHODS\\
- Text classification\\
automatically categorizing textual data


%%%%%%%%%%%%%%%%%%%%%%%%%%%%%%%%%%%%%%%%%%%%%%%%%%%

\clearpage
\section{Specification of requirements}

\clearpage{}
\section{Implementation}

\subsection{Dataset}


\clearpage{}
\section{Conclusion}


\clearpage
\thispagestyle{empty}
\mbox{}
\clearpage





%%%%%%%%%%%%%%%%%%%%%%%%%%%%%%%%%%%%%%%%%%%%%%%%%%%

%\clearpage %\cite{Al-Hassan2024a} %\cite{Wang2024} %%\nocite{DeBiasio2024}
%\nocite{5197422} %\nocite{NILASHI2018507} %\printbibliography{}


\clearpage 
\normalsize 
\bibliographystyle{unsrt} 
\bibliography{literature} 
\nocite{*}

\end{document}
